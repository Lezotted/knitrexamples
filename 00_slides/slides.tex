%%%%%%%%%%%%%%%%%%%%%%%%%%%%%%%%%%%%%%%%%%%%%%%%%%%%%%%%%%%%%%%%%%%%%%%%%%%%%%%%
% file:   slides.Rnw
% Author: Peter DeWitt, peter.dewitt@ucdenver.edu
% 
% These slides are based on the knitr-manual written by Yihui Xie.  Original
% .Rnw file can be accessed via R 
% system.file("examples", "knitr-manual.Rnw", package = "knitr")
%
% change log:
%  9 Nov 2013 - file created
% 17 Jan 2014 - fixed some spelling and grammar errors added an empty frame for 
%               notes about git to be placed.
% 23 Jan 2014 - editing this file to get it ready for presentation at UCD
% 25 Jan 2014 - added to xkcd comics for amusement
%%%%%%%%%%%%%%%%%%%%%%%%%%%%%%%%%%%%%%%%%%%%%%%%%%%%%%%%%%%%%%%%%%%%%%%%%%%%%%%%

% preamble %{{{
\documentclass[t]{beamer}\usepackage[]{graphicx}\usepackage[]{color}
%% maxwidth is the original width if it is less than linewidth
%% otherwise use linewidth (to make sure the graphics do not exceed the margin)
\makeatletter
\def\maxwidth{ %
  \ifdim\Gin@nat@width>\linewidth
    \linewidth
  \else
    \Gin@nat@width
  \fi
}
\makeatother

\definecolor{fgcolor}{rgb}{0.345, 0.345, 0.345}
\newcommand{\hlnum}[1]{\textcolor[rgb]{0.686,0.059,0.569}{#1}}%
\newcommand{\hlstr}[1]{\textcolor[rgb]{0.192,0.494,0.8}{#1}}%
\newcommand{\hlcom}[1]{\textcolor[rgb]{0.678,0.584,0.686}{\textit{#1}}}%
\newcommand{\hlopt}[1]{\textcolor[rgb]{0,0,0}{#1}}%
\newcommand{\hlstd}[1]{\textcolor[rgb]{0.345,0.345,0.345}{#1}}%
\newcommand{\hlkwa}[1]{\textcolor[rgb]{0.161,0.373,0.58}{\textbf{#1}}}%
\newcommand{\hlkwb}[1]{\textcolor[rgb]{0.69,0.353,0.396}{#1}}%
\newcommand{\hlkwc}[1]{\textcolor[rgb]{0.333,0.667,0.333}{#1}}%
\newcommand{\hlkwd}[1]{\textcolor[rgb]{0.737,0.353,0.396}{\textbf{#1}}}%

\usepackage{framed}
\makeatletter
\newenvironment{kframe}{%
 \def\at@end@of@kframe{}%
 \ifinner\ifhmode%
  \def\at@end@of@kframe{\end{minipage}}%
  \begin{minipage}{\columnwidth}%
 \fi\fi%
 \def\FrameCommand##1{\hskip\@totalleftmargin \hskip-\fboxsep
 \colorbox{shadecolor}{##1}\hskip-\fboxsep
     % There is no \\@totalrightmargin, so:
     \hskip-\linewidth \hskip-\@totalleftmargin \hskip\columnwidth}%
 \MakeFramed {\advance\hsize-\width
   \@totalleftmargin\z@ \linewidth\hsize
   \@setminipage}}%
 {\par\unskip\endMakeFramed%
 \at@end@of@kframe}
\makeatother

\definecolor{shadecolor}{rgb}{.97, .97, .97}
\definecolor{messagecolor}{rgb}{0, 0, 0}
\definecolor{warningcolor}{rgb}{1, 0, 1}
\definecolor{errorcolor}{rgb}{1, 0, 0}
\newenvironment{knitrout}{}{} % an empty environment to be redefined in TeX

\usepackage{alltt}

\usepackage{beamerthemesplit}
\usepackage{graphics,graphicx}
\usepackage{ulem,color}
\usepackage{wrapfig}
\usepackage{rotate}
\usepackage{verbatim}
\usepackage{array} 
\usepackage{ctable}
\usepackage{url}

\usetheme{Darmstadt}
\usecolortheme{beetle}
\usefonttheme{serif}
\setbeamercolor{framesubtitle}{fg=white}

\title[Reproducible Research]{Reproducible Research}

% \\Using Literate Programming\\and\\version control}
\subtitle{A brief overview of\\Literate Programming via R and {\tt
knitr}\\and\\Version control via git}
\author{Peter DeWitt}
\institute{University of Colorado Anschutz Medical Campus 
 
Department of Biostatistics}

\date{29 January 2014}




%}}}

%%%%%%%%%%%%%%%%%%%%%%%%%%%%%%%%%%%%%%%%%%%%%%%%%%%%%%%%%%%%%%%%%%%%%%%%%%%%%%%
\IfFileExists{upquote.sty}{\usepackage{upquote}}{} 

\begin{document}
%{{{
\begin{frame}[fragile]
  \titlepage
\end{frame}

% \section[Outline]*{} 
\section*{} 
\begin{frame}
  \frametitle{Presentation Outline}
  \tableofcontents
\end{frame}

\begin{frame}
  \frametitle{A little humor}
  \begin{minipage}[m]{0.58\textwidth}
    \includegraphics[width=0.98\textwidth]{figure/file_extensions.png}
  \end{minipage}
  ~
  \begin{minipage}[m]{0.38\textwidth}
    \begin{center} 
       www.xkcd.com/1301
    \end{center}

       Alt Text: I have never been lied to by data in a .txt file which has been
       hand-aligned.

       \vspace{0.5in}

       The image itself is a .png
  \end{minipage} 
\end{frame}

\begin{frame}
  \frametitle{A little humor}
  \begin{minipage}[m]{0.58\textwidth}
    \includegraphics[width=0.98\textwidth]{figure/automation.png}
  \end{minipage}
  ~
  \begin{minipage}[m]{0.38\textwidth}
      
      \begin{center}
        www.xkcd.com/1319
      \end{center}

      Alt Text: `Automating' comes from the roots `auto; meaning `self-', and
      `mating', meaning `screwing'.
  \end{minipage} 
\end{frame}
%}}}

\section{Introduction}
%{{{
\begin{frame}[fragile] 
  \frametitle{Literate Programming}
  \begin{itemize}
    \item Originally intended for software development
    \item Mix source code (for the computer) and documentation (for the humans)
      together
    \item Sweave built on this paradigm but with a different focus:
      reproducible data analysis and statistical reports.
    \item {\tt knitr}, developed by Yihui Xie, expands on the concept of Sweave.
    \item `designed to give the user access to every part of the process of
      dealing with a literate programming document'
    \item package homepage \url{http://yihui.name/knitr/}
  \end{itemize}
\end{frame}

\begin{frame} 
  \frametitle{Dynamic Report Writing}
  \begin{itemize}
    \item Pros: 
      \begin{itemize}
        \item Reproducible reports 
        \item Contextual commenting 
        \item {\tt knitr} is flexible enough to allow for multiple analysis languages
          and multiple markup languages.
      \end{itemize}
    \item Cons:
      \begin{itemize}
        \item Not all analysis languages are ideally suited for this paradigm.
        \item Collaborations with others using WYSIWYG editors requires some
          additional work and breaks automation/reproducibility.
      \end{itemize}
    \item Additional Tools: Version control, e.g., git or svn.
    \item[]
    \item Every project is different; use the tools that will work best.
  \end{itemize}
\end{frame}

%}}}

\section{Hello World}
%{{{
\begin{frame}
  \frametitle{Why {\tt knitr}?}
  \begin{itemize}
    \item Incorporate both the analysis code and the manuscript writing into one
      file.
    \item Contextually commented analysis code.
    \item Fully documented and reproducible reports.
  \end{itemize}
\end{frame}

\begin{frame}[fragile]
  \frametitle{Example: Code for the next two frames}
  % The following is the \LaTeX\ and R code responsible for the following slide.
 
  \footnotesize
\verb;\begin{frame}[fragile];
\begin{verbatim}<<"cars", fig.width = 3.5, fig.height = 3.25, results = "asis">>=
fit <- lm(dist ~ speed, data = cars) 
latex(cbind(coef(fit), confint(fit)), 
      file = "", title = "", ctable = TRUE, 
      caption = "Regression Estimates", 
      digits = 3, colhead = c("Est", "LCL", "UCL"))
@ 
\end{verbatim}

{\tt 
The expected stopping distance for a car during the 1920s 
increased by $\backslash$Sexpr\{round(coef(fit)[2], 2)\} 
feet for every additional mph increase in speed.} 

\verb;\end{frame};

\verb;\begin{frame}[fragile];
  \begin{verbatim}<<"cars_plot", fig.width = 3.5, fig.height = 2.75>>=
qplot(speed, dist, data = cars) + geom_smooth(method = "lm")
@
\end{verbatim}
\verb;\end{frame};
\end{frame}

\begin{frame}[fragile]
\begin{kframe}
\begin{alltt}
\hlstd{fit} \hlkwb{<-} \hlkwd{lm}\hlstd{(dist} \hlopt{~} \hlstd{speed,} \hlkwc{data} \hlstd{= cars)}
\hlkwd{latex}\hlstd{(}\hlkwd{cbind}\hlstd{(}\hlkwd{coef}\hlstd{(fit),} \hlkwd{confint}\hlstd{(fit)),}
      \hlkwc{file} \hlstd{=} \hlstr{""}\hlstd{,} \hlkwc{title} \hlstd{=} \hlstr{""}\hlstd{,} \hlkwc{ctable} \hlstd{=} \hlnum{TRUE}\hlstd{,}
      \hlkwc{caption} \hlstd{=} \hlstr{"Regression Estimates"}\hlstd{,}
      \hlkwc{digits} \hlstd{=} \hlnum{3}\hlstd{,} \hlkwc{colhead} \hlstd{=} \hlkwd{c}\hlstd{(}\hlstr{"Est"}\hlstd{,} \hlstr{"LCL"}\hlstd{,} \hlstr{"UCL"}\hlstd{))}
\end{alltt}
\end{kframe}% latex.default(cbind(coef(fit), confint(fit)), file = "", title = "",      ctable = TRUE, caption = "Regression Estimates", digits = 3,      colhead = c("Est", "LCL", "UCL")) 
%
\ctable[caption={Regression Estimates},label=,pos=!tbp,]{lrrr}{}{\FL
\multicolumn{1}{l}{}&\multicolumn{1}{c}{Est}&\multicolumn{1}{c}{LCL}&\multicolumn{1}{c}{UCL}\ML
(Intercept)&$-17.58$&$-31.2$&$-3.99$\NN
speed&$  3.93$&$  3.1$&$ 4.77$\LL
}


  The expected stopping distance for a car during the 1920s increased by
  3.93 feet for every additional mph increase in
  speed.
\end{frame}

\begin{frame}[fragile]
  \frametitle{}
\begin{knitrout}\footnotesize
\definecolor{shadecolor}{rgb}{0.969, 0.969, 0.969}\color{fgcolor}\begin{kframe}
\begin{alltt}
\hlkwd{qplot}\hlstd{(speed, dist,} \hlkwc{data} \hlstd{= cars)} \hlopt{+} \hlkwd{geom_smooth}\hlstd{(}\hlkwc{method} \hlstd{=} \hlstr{"lm"}\hlstd{)}
\end{alltt}
\end{kframe}

{\centering \includegraphics[width=\maxwidth]{figure/cars_plot} 

}



\end{knitrout}

\end{frame}

%}}}

\section{Design}
%{{{
\begin{frame}[fragile]
  \frametitle{How does {\tt knitr} work?}
  \begin{itemize}
    \item One input file with 
      \begin{itemize}
        \item an analysis language (R, Python, awk, SAS, \ldots) and 
        \item an output markup language (\LaTeX, html, Markdown, \ldots)
      \end{itemize}

    \item {\tt knitr} determines the appropriate set of patterns (regular
      expression to extract analysis language and options from the input file)

    \item The input file is knitted\ldots, analysis language is evaluated and
      the appropriate output markup results are placed into a .tex, .html, .md,
      \ldots, file.

    \item Final document (the .tex, .html, .md, \ldots)  is ready for release or
      post processing as needed.

  \end{itemize}
\end{frame} 

%}}}

\section{Options}
%{{{
\subsection{Chunk Options}
\begin{frame}[fragile]
  \frametitle{Customizing the behavior of {\tt knitr}}
  For Rnw files:

  \begin{verbatim}<<"chunk_label", echo = FALSE, results = "asis">>=
@
  \end{verbatim}
  \begin{itemize}
    \item Chunk options must be a single line; no line breaks.  
    \item Options must be valid R expressions.
    \item Chunk options can be specified for each individual chunk.
    \item Global options are set via \verb;opts_chunk$set();
  \end{itemize}
\end{frame}

\begin{frame}
  \frametitle{Chunk Options}
  Full details for all the chunk options see
  \url{http://yihui.name/knitr/options}
  \begin{itemize}
    \item Code Evaluation
    \item How to display results
    \item Code Decoration
    \item Cache
    \item Plots
    \item Animation
    \item Chunk References
    \item Child Documents
    \item Language Engines
    \item Extracting source code
  \end{itemize}
\end{frame}

\subsection{Language Options}
\begin{frame}
  \frametitle{Input/Analysis Language}
  \begin{itemize}
    \item R is the `default' language for analysis 
    \item Other options are available, including SAS.  The chunk option `engine'
      allows for different languages to be used.

      \begin{quote}
        engine: ('R'; character) the language name of the code chunk; currently
        other possible values are 'python' and 'awk'/'gawk'; the object
        knit\_engines in this package can be used to set up engines for other
        languages\footnote{http://yihui.name/knitr/options}
      \end{quote}

    \item Pick the right language for the job.  R is great, but every now and
      then SAS would be preferable.  
  \end{itemize}
\end{frame}

\begin{frame}
  \frametitle{Markup Language}
  \framesubtitle{html}
  \begin{itemize}
    \item Pros: 
      \begin{itemize}
        \item easy to send to others, 
        \item comments in html, 
        \item great for tutorials or anything that will be published online.
      \end{itemize}
    \item Cons: 
      \begin{itemize}
        \item Clunky (in my opinion), 
        \item not apt for large data analysis reports.
      \end{itemize}

    \item html code can be placed natively in markdown, ergo, markdown has
      supplanted html.
  \end{itemize}
\end{frame}

\begin{frame}
  \frametitle{Markup Language}
  \framesubtitle{Markdown}
  \begin{itemize}
    \item Pros: 
      \begin{itemize}
        \item Easy to learn 
        \item Simple and versatile
        \item Growing user community  
        \item Via pandoc, easy to convert to many other file formats such as
          \LaTeX, html, or .docx.
      \end{itemize}
    \item Cons: 
      \begin{itemize}
        \item no native comments.  
        \item `too minimal'
        \item pandoc may require additional scripting to format output as needed
          for .docx or other file types.
      \end{itemize}
  \end{itemize}
\end{frame}

\begin{frame}
  \frametitle{Markup Language}
  \framesubtitle{\LaTeX}
  \begin{itemize}
    \item Pros: 
      \begin{itemize}
        \item Intended use: technical report writing and typesetting.
        \item Comments.
        \item Many tools exist for formating R output well in \LaTeX\ files.
        \item Cross referencing, citations.
        \item pdflatex produces a file that is almost platform independent.
      \end{itemize}

    \item Cons:
      \begin{itemize}
        \item Difficulties will be encountered  when collaborating with
          Microsoft Office users.
        \item R is small,  Tex Live, MacTeX, and proTeX are not.
        \item Steepest learning curve
      \end{itemize}
  \end{itemize} 
\end{frame}

\begin{frame}
  \frametitle{Markup Language}
  \framesubtitle{My Preferences}
  \begin{itemize}
    \item \LaTeX\ is my preferred markup language for data analysis reports and
      presentations (via beamer).

    \item Markdown is my preferred markup language for developing a web page.
      Easier to work in than html and more flexible.

  \end{itemize} 

  \[ 
  \text{.Rmd} \stackrel{\text{knitr}}{\rightarrow} 
  \text{.md } \stackrel{\text{pandoc}}{\rightarrow} 
  \begin{cases} 
        \text{.html} & \text{okay} \\
        \text{.doc(x)} & \text{might require VB scripting} \\
        \text{.pdf}  & \text{Use \LaTeX} 
      \end{cases}\]

\end{frame}

\subsection{Collaborations}
\begin{frame}[fragile]
  \frametitle{That's great, but \ldots}

  \begin{itemize}
    \item \LaTeX\ pdfs are not always easy to transfer to MS Word
    \item Local html pages are easy to break.
    \item Misconceptions about viewing local files versus remote files in a web
      browser.
    \item .doc(x) are `accidentally' editable.  %So much for reproducible research.
  \end{itemize}

  \begin{itemize}
    \item What about Google docs?
    \item What about track changes in Office?
  \end{itemize}

\end{frame}

%}}}

\section{Editors} 
%{{{
\begin{frame}
  \frametitle{Suggested Development Environments}

  \begin{itemize}
    \item I prefer the Vim editor and with the vim-r-plugin.  
      \begin{itemize}
        \item Vim Editor: \url{www.vim.org/index.php}
        \item vim-r-plugin: \url{www.vim.org/scripts/script.php?script\_id=2628} 
        \item Completely customizable.
      \end{itemize}

    \item If you are not a Vim or Emacs user, then RStudio is the \emph{premier}
      R IDE for you.
      \begin{itemize}
        \item Download and info: \url{www.rstudio.com}

        \item Built in tools for version control, projects, knitting\ldots
      \end{itemize}


    \item RStudio is the better R development environment.  
    \item Vim is a better text editor.

    \item A pseudo WYSIWYG editor for \LaTeX\ which will work well with 
      {\tt knitr} is LyX.
      
  \end{itemize}
\end{frame}
%}}}

\section{Examples}
%{{{
\begin{frame}
  \frametitle{Reproducible examples}
  These slides, and the following examples, can be downloaded/cloned, from
  \url{https://github.com/dewittpe/knitrexamples}

  \begin{itemize}
    \item A simple data analysis report using R, and three different markup
      languages, \LaTeX, html, and Markdown.
    \item An example of using SAS within {\tt knitr}.
    \item An example of a more complex and extensive data analysis report.
  \end{itemize}
\end{frame}
%}}}

\section{git}
%{{{
\begin{frame}[fragile]
  \frametitle{Version Control with git}
  \begin{itemize}
    \item Distributed version control: everything is local and can be synced
      with a remote server.
    \item Simple to use.  
      \begin{itemize}
        \item Built in GUI with RStudio
        \item Only need six commands to be able to use git.
          \begin{enumerate}
            \item git status
            \item git log
            \item git add
            \item git commit
            \item git pull -- only if working with a remote
            \item git push -- only if working with a remote
          \end{enumerate}
        \item Other very helpful commands: tag, fetch, merge, mergetool, and
          branch
      \end{itemize}
  \end{itemize}
\end{frame}

\begin{frame}
  \frametitle{Do you need a server?}
  \begin{itemize}
    \item The repositories are local, so servers are not needed.
    \item Servers to host a central repository are helpful for collaborations.
    \item Read/Write permissions can be set for different users.
    \item There are options to put a git enterprise on a server behind our
      firewall, but it's expensive.
    \item github.com has free public repos, pay for private repos
    \item bitbucket.org has free public and private repos.
    \item[]
    \item Visit \url{http://git-scm.com/book} for documentation on git.
  \end{itemize}
\end{frame}

% \begin{frame}
%   \frametitle{More on git}
%   \begin{itemize}
%     \item Fully introducing git would take more time than is available right
%       now.
%   \end{itemize}
% \end{frame}
%}}}

%%%%%%%%%%%%%%%%%%%%%%%%%%%%%%%%%%%%%%%%%%%%%%%%%%%%%%%%%%%%%%%%%%%%%%%%%%%%%%%

\end{document} 
%%%%%%%%%%%%%%%%%%%
%%% End of File %%%
%%%%%%%%%%%%%%%%%%%

